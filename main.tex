% This is a LaTeX template
% for preparing documents for ITTMM conference

\documentclass[60x84/16,8pt]{ittmm}

% Убедительная просьба к авторам не редактировать файл definition.tex
\input{definition}

\begin{document}

% Укажите индекс УДК, соответствующий Вашей работе.
\udc{004.4}

\title{Шаблон оформления рукописи доклада
  на конференцию <<ITTMM>>}

\author[1,2]{А. Б. Первый}
\author[1]{В. Г. Второй}

\address[1]{Кафедра прикладной информатики и теории вероятностей,\\
  Российский университет дружбы народов,\\
  ул. Миклухо-Маклая, д.6, Москва, Россия, 117198}
\address[2]{Лаборатория информационных технологий,\\
Объединённый институт ядерных исследований,\\
ул. Жолио-Кюри 6, Дубна, Московская область, Россия, 141980}

\email{\url{first@rudn.university}, \url{second@rudn.university}}

\begin{abstract}
Разместите здесь аннотацию на русском языке (150--250 слов).
\end{abstract}

\keywords{компьютерные науки, информационные технологии, проведение конференции}

\thanks{Убедительная просьба к авторам не вводить свои макроопределения.}

% \thanks{Рукопись должна содержать УДК, который рекомендуется брать из
%   следующего источника: \url{http://www.mathnet.ru/udc.pdf}.}

\alttitle{ITTMM Conference Thesis Template}

\altauthor[1,2]{A. B. First}
\altauthor[1]{C. D. Second}

\altaddress[1]{Department of Applied Probability and Informatics\\
Peoples' Friendship University of Russia\\
Miklukho-Maklaya str. 6, Moscow, 117198, Russia}
\altaddress[2]{Laboratory of Information Technologies\\
Joint Institute for Nuclear Research\\
Joliot-Curie 6, Dubna, Moscow region, 141980, Russia}

\begin{altabstract}
Place here short abstract in English (between 150 and 250 words).
\end{altabstract}

\altkeywords{computer science, information technologies, conference proceedings}

\maketitle

\section{Введение}
\label{sec:intro}

Во введении обычно излагают основные сведения о поставленной задаче, о
её месте в области научных знаний и их приложений.  Здесь, по
возможности, должен содержаться краткий обзор современного состояния
данной проблемы (критический анализ научной литературы и заключение по
этому анализу), а также краткая историко"=библиографическая справка по
проблемам, близким к решаемой задаче.  Здесь же формулируются цели и
задачи исследования, ставится конкретная математическая задача и
методы ее решения, отмечаются элементы новизны и практической
ценности.

Каждый автор имеет право на участие не более, чем в трёх докладах.  В
одном докладе не рекомендуется участие более четырёх авторов.

Структура тезиса, предложенная в данном шаблоне, имеет
рекомендательное значение. 

\section{Основная часть} 
\label{sec:base-section}

Основная часть работы должна отражать поэтапное подробное решение
поставленной задачи и может содержать несколько разделов. Здесь
проводятся доказательства и решения выдвинутых положений и задач,
рассматриваются методы их решения, приводится наглядный иллюстративный
материал в виде графиков, таблиц, диаграмм и т.~д.

По требованиям организационного комитета конференции объём одной
представляемой рукописи не должен превышать 3-х страниц. Авторы
обязаны предъявлять повышенные требования к изложению и языку
рукописи, а также подготовке иллюстративного и табличного материалов.
Рукопись представляется на русском языке.  Рекомендуется безличная
форма изложения.

При оформлении рекомендуется пользоваться стандартными окружениями
макропакета \LaTeXe.

<<Ссылочный аппарат>> на формулы реализуется с помощью команд
\verb"\label" и \verb"\eqref".\footnote{Нумероваться будут только те
  формулы, на которые ссылка оформлена с помощью этих команд.}  В
качестве примера приведём формулу
\begin{equation}
a^n+b^n=c^n
\label{eq:Fermat's_Last_Theorem}
\end{equation}
и ссылку на неё \eqref{eq:Fermat's_Last_Theorem}.

Рисунки в рукопись вставляются стандартными средствами \LaTeXe.  В
качестве форматов рисунков рекомендуется использовать файлы типа
\texttt{eps} или \texttt{pdf}, изображение должно быть качественным и
векторным. Разрешение растровой графики должно быть не менее 600 dpi
(лучше 1200 dpi). Шрифт на рисунках не должен быть менее 6 пунктов.
Каждый рисунок должен быть подписан, для этого
используется команда \verb"\caption". 
Как пример см. рис.~\ref{fig:logo}.

\begin{figure}
  \centering
  \includegraphics[width=0.2\linewidth]{embl}
  \caption{Эмблема}
  \label{fig:logo}
\end{figure}

Ниже (см. табл.~\ref{tab:sampletable}) представлен вариант таблицы с
заголовком оформленным с помощью \verb"\caption".

\begin{table}
  \centering
  \caption{Пример небольшой таблицы}
  \label{tab:sampletable}
  \begin{tabular}{|c|c|c|c|c|}
    \hline
    Номер & $X$ & $Y$ & $R$ & Цвет\\
    \hline
    1 &     100  &  170 & 30 & красный\\
    2 &     100  &  90      & 60 & жёлтый\\
    3 &     230  &  250     & 50 & синий\\
    4 &     130  &  240 & 60 & зелёный\\
    5 & 300  &      130 & 30 & зелёный\\
    6 &     200  &  150     & 90 & красный\\
    \hline
  \end{tabular}
\end{table}

Для ссылки на источники необходимо использовать команду \verb"\cite".

Литература может формироваться либо с помощью программы \verb"bibtex",
либо с помощью окружения \verb"thebibliography".
При формировании списка литературы просьба использовать стандарт
ГОСТ~Р7.0.5-2008. Примеры цитирования книги~\cite{mathtensor,
  jones-fogelin:tcqd}, раздела в книге~\cite{Muller2006},
статьи~\cite{Arduengo1991, Booth1962},
материалов конференции~\cite{Hope2005}.

На все источники в списке литературы должны быть ссылки.

\section{Заключение}

Заключение является неотъемлемой частью любой работы. 

Оно должно содержать краткие выводы по результатам исследования,
отражающие новизну и практическую значимость работы, предложения по
использованию ее результатов, оценку её эффективности и качества.


\begin{acknowledgments}
  Работа частично поддержана грантом РФФИ
  \textnumero~16-01-20379.\footnote{Этот раздел статьи может
    отсутствовать.  В него рекомендуется добавлять сведения о
    финансировании работы и выражать благодарности персонам.}
\end{acknowledgments}


% \begin{thebibliography}{99}

% \bibitem{mathtensor}
% L.~Parker, S.~M. Christensen, MathTensor: a system for doing tensor analysis by
%   computer, Addison-Wesley, 1994.

% \bibitem{jones-fogelin:tcqd}
% W.~T. Jones, R.~J. Fogelin, The Twentieth Century to Quine and Derrida, A
%   History of Western Philosophy, Harcourt Brace College Publishers, 1997.

% \bibitem{Muller2006}
% G.~M. Sheldrick, A Short History of SHELXL, International Union of
%   Crystallography and Oxford University Press, 2006.

% \bibitem{Arduengo1991}
% A.~J. Arduengo, III, R.~L. Harlow, M.~Kline, A stable crystalline carbene,
%   J.~Am. Chem. Soc. 113~(1)  361--363.
% \newblock \href {http://dx.doi.org/10.1021/ja00001a054}
%   {\path{doi:10.1021/ja00001a054}}.

% \bibitem{Booth1962}
% G.~Booth, J.~Chatt, The reactions of carbon monoxide and nitric oxide with
%   tertiary phosphine complexes of iron({II}), cobalt({II}), and nickel({II}),
%   J.~Chem. Soc.  2099--2106. \href {http://dx.doi.org/10.1039/JR9620002099}
%   {\path{doi:10.1039/JR9620002099}}.

% \bibitem{Hope2005}
% E.~Hope, J.~Bennett, A.~Stuart, Fluorous zirconium phosphonates: novel
%   inorganic supports for catalysis, in: Pacifichem (International Chemical
%   Congress of Pacific Basin Societies), no. 961, Pacific Basin Chemical
%   Societies.

% \end{thebibliography}


%% Возможно использовать bibtex.
\bibliographystyle{ugost2008l}
\bibliography{main}

\makealttitle

\end{document}
